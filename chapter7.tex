\section{Amerikanische Optionen}

\begin{karte}{Modellannahmen}
\begin{itemize}
    \item Der Finanzmarkt sei arbitragefrei und vollständig. 
    \item Eine amerikanische Option wird repräsentiert durch einen \((\F_t)\)-adaptierten 
    stochastischen Prozess \((H_t)\).
    \item Ein Ausübungszeitpunkt wird durch eine \((\F_t)\)-Stoppzeit 
    \(\abb{\tau}{\Omega}{\set{0,\ldots, T}}\) beschrieben. 
    \item Verwendet der Käufer der Option den Ausübungszeitpunkt \(\tau\), dann ist 
    \[ H_\tau(\omega) = \sum_{t=0}^T H_t(\omega) 1_{\set{\tau(\omega) = t}} = H_{\tau(\omega)}(\omega) \]
    die Auszahlung der Option.
\end{itemize}
\end{karte}

\begin{karte}{Amerikanische Put-Option, Bermuda-Option}
Amerikanische Put-Option:
\begin{itemize}
    \item Es ist \(H_t = (K - S_t)^+\). 
    \item Möglicher Ausübungszeitpunkt \(\tau = \min \set{t \in \N: S_t \leq K} \wedge T\).
\end{itemize}

Bermuda-Option:
\begin{itemize}
    \item Bei einer Bermuda-Option hat der Käufer das Recht, zu bestimmten Zeitpunkten 
    \(\mathcal{T} := \set{t_1, \ldots, t_k} \subset \set{0, \ldots, T}\) auszuüben.
    \item Ist \(\mathcal{T} = \set{T}\), dann ist es eine europäische Option. 
    \item Ist \(\mathcal{T} = \set{0,\ldots, T}\), dann ist es eine amerikanische Option. 
    \item Definiert man die Wertpapierkurse nur auf \(\mathcal{T}\), so kann man die Bewertung auf 
    die amerikanische Option zurückführen.
\end{itemize}
\end{karte}

\begin{karte}{Stoppprobleme}
Sei \((X_t)\) ein \((\F_t)\)-adaptierter stochastischer Prozess mit \(\E \abs{X_t} < \infty \forall t\in\N_0\).

Zu lösen ist das Stoppproblem 
\[ \sup_{\tau \leq T} \E[X_\tau], \]
wobei das Supremum über alle \((\F_t)\)-Stoppzeiten \(\tau\) genommen wird 
mit \(\P(\tau \leq T) = 1\).
\end{karte}

\begin{karte}{Snell-Einhüllende}
Ein stochastischer Prozess \((Z_t)\), definiert durch 
\begin{align*}
    Z_T &:= X_T, \\
    Z_t &:= \max\set{ X_t, \E[Z_{t+1} | \F_t], }
\end{align*}
heißt Snell-Einhüllende von \((X_t)\).

Die Snell-Einhüllende \((Z_t)\) von \((X_t)\) ist ein Supermartingal und das kleinste 
Supermartingal, welches \((X_t)\) dominiert, d. h. \(Z_t \geq X_t\) für alle \(t=0,\ldots, T\).

Sei \(\tau^* := \inf \set{t\geq 0: Z_t = X_t}\). Dann ist 
\(\tau^* \) eine Stoppzeit, \(\tau^* \leq T\) und der gestoppte Prozess 
\((Z_{t\wedge \tau^*})\) ist ein Martingal.
\end{karte}

\begin{karte}{Lösung Stoppproblem}
Die Stoppzeit \(\tau^* = \inf \set{t\geq 0: Z_t = X_t}\) löst das Stoppproblem 
und es ist 
\[ Z_0 = \E[X_{\tau^*}] = \sup_{\tau \leq T} \E[X_\tau]. \]
\end{karte}

\begin{karte}{Preis einer amerikanischen Option}
\[ \pi^A(H) := \sup_{\tau \leq T} \E_\Q[\frac{H_\tau}{B_\tau}],\]
wobei das Supremum über alle Stoppzeiten \(\tau\) mit \(\tau \leq T\) genommen wird.

Sei \((Z_t)\) die Snell-Einhüllende von \((H_t B_t^{-1})\) und 
\[ \tau^* = \inf \set{t\geq 0: Z_t = \frac{H_t}{B_t}}. \]
Dann ist der Preis einer amerikanischen Option gegeben durch 
\[ \pi^A(H) = Z_0 = \sup_{\tau \leq T} \E_\Q [\frac{H_\tau}{B_\tau}] \]
und \(\tau^*\) ist ein optimaler Ausübungszeitpunkt, d. h. 
\[ \E_\Q[\frac{H_{\tau^*}}{B_{\tau^*}}] = \sup_{\tau \leq T} \E_\Q [\frac{H_\tau}{B_\tau}]. \]
\end{karte}

\begin{karte}{Preis einer amerikanischen Option zur Zeit \(t\)}
Für den Preis \(\pi_t^A(H)\) zur Zeit \(t\) müssen wir folgendes Stoppproblem lösen: 
\[ \pi_t^A(H) := B_t \sup_{t\leq \tau \leq T} \E_\Q[\frac{H_\tau}{B_\tau} | \F_t]. \]
Nach Konstruktion der Snell-Einhüllenden \((Z_t)\) gilt: \(\pi_t^A(H) = B_t Z_t\).
\end{karte}

\begin{karte}{Spezialfall \(H_t/B_t\) Submartingal}
Ist \((H_t B_t^{-1})\) ein \(\Q\)-Submartingal, dann gilt 
\[ \pi_t^A(H) = \pi_t(H_T) \]
und \(\tau^* = T\) ist ein optimaler Ausübungszeitpunkt. 
Dabei ist \(\pi_t(H_T)\) der Preis der europäischen Option \(H_T\) zur Zeit \(t\).
\end{karte}

\begin{karte}{Preisungleichungen}
Sei \(C_t^A\) der Preis einer amerikanischen Call-Option mit Basispreis \(K\) zur Zeit \(t\)
und \(P_t^A\) der Preis einer amerikanischen Put-Option mit Basispreis \(K\) zur Zeit \(t\).

Dann gilt die folgende Ungleichung: 
\[ S_t - K \leq C_t^A - P_t^A \leq S_t - K \frac{B_t}{B_T}. \]
\end{karte}

\begin{karte}{Bewertung amerikanischer Optionen im CRR-Modell}
Betrachte Spezialfall \(H_t = h(S_t)\). Snell-Einhüllenden von \((H_t B_t^{-1})\): 
\begin{align*}
    Z_T &= \frac{h(S_T)}{B_T} = \frac{h(S_T)}{(1+r)^T} =: Z_T(S_T), \\
    Z_{T-1} &= \max\set{ \frac{H(S_{T-1})}{B_{T-1}}, \E_\Q[Z_T | \F_{T-1}] } \\
    &= \max \set{ \frac{h(S_{T-1})}{(1+r)^{T-1}}, q Z_T(S_{T-1} u) + (1-q) Z_T(S_{T-1}d) } =: Z_{T-1}(S_{T-1}) \\
    Z_t &= \max\set{ \frac{h(S_t)}{(1+r)^t}, q Z_{t+1}(S_t u) + (1-q) Z_{t+1}(S_t d) } =: Z_t(S_t), \\
    Z_0 &= \max \set{ h(S_0), q Z_1 (S_0 u) + (1-q) Z_1(S_0 d) }.
\end{align*}

Für den Preisprozess gilt \(p_0(S_0) = \pi^A(H)\). 
\[ \pi_t^A(H) = B_t Z_t = (1+r)^t Z_t(S_t) =: p_t(S_t). \]
\begin{align*}
    p_T(S_T) &= h(S_T), \\
    p_t(S_t) &= \max \set{ h(S_t), \frac{1}{1+r} (qp_{t+1}(S_t u) + (1-q) p_{t+1}(S_t d))}.
\end{align*}
\end{karte}

\subsection{Hedging}

\begin{karte}{Hedging von amerikanischen Optionen}
Eine selbstfinanzierende Handelsstrategie \(\varphi\) heißt Hedging-Strategie 
(aus Sicht des Verkäufers) für eine amerikanische Option \((H_t)\), falls gilt: 
\[ V_t^\varphi \geq H_t. \]
Sei \((H_t)\) eine amerikanische Option. Dann gibt es eine Hedging-Strategie \(\varphi\) 
mit 
\[ V_0^\varphi = \pi^A(H) = \sup_\tau \E_\Q[\frac{H_\tau}{B_\tau}] = Z_0, \]
wobei \((Z_t)\) die Snell-Einhüllende von \((H_t B_t^{-1})\) ist.

Ist \(\varphi\) eine beliebige selbstfinanzierende Hedging-Strategie für \((H_t)\), 
so gilt, dass \(( \frac{V_t^\varphi}{B_t} )\) ein \(\Q\)-Martingal mit 
\(\frac{V_t^\varphi}{B_t} \geq \frac{H_t}{B_t}\) ist, damit \(Z_t \leq \frac{V_t^\varphi}{B_t}\). 

Insgesamt folgt 
\[ \pi^A(H) = \inf \set{ V_0^\varphi: \varphi \text{ ist eine selbstfinanz. Hedging-Strategie für } (H_t) }. \]
\end{karte}

\begin{karte}{Hedging amerikanischer Optionen im CRR-Modell}
Wir betrachten das CRR-Modell und den Spezialfall \(H_t = h(S_t)\).
Der Preisprozess sei \((p_t)\).
Wir wollen nun eine Hedging-Strategie \(\varphi = (\alpha, \beta)\) für \((H_t)\) 
bestimmen mit \(V_0^\varphi = Z_0\). Sei für \(t=0,\ldots, T-1\): 
\begin{align*}
    \alpha_t &:= \frac{p_{t+1}(uS_t) - p_{t+1}(d S_t)}{(u-d)S_t} \\
    c_t &:= p_t(S_t) - \frac{1}{1+r} (q p_{t+1} ( u S_t) + (1-q) p_{t+1}(d S_t)).
\end{align*}
Interpretation: \((\alpha_t)\) ist die Investitionsstrategie in das risikobehaftete Wertpapier, 
\((c_t)\) ist die Konsumstrategie. 
Vermögensprozess \((V_t)\): 
\begin{align*}
    V_0 &:= p_0 (S_0) = Z_0 \\
    V_{t+1} &:= \alpha_t S_{t+1} + (1+r) ( V_t - c_t - \alpha_t S_t).
\end{align*}
Sei \(\tau^* := \min\set{t\leq T: p_t(S_t) = h(S_t)}\) der optimale Ausübungszeitpunkt. 
Dann gilt für \(t < \tau^*\), dass \(c_t = 0\) und \(V_t\) ist das Vermögen der Strategie \((\alpha_t)\) 
zur Zeit \(t\).

Es gilt \(V_t = p_t(S_t)\) für \(t=0,\ldots, T\), d. h. \((\alpha_t)\) ist eine 
Hedging-Strategie für die amerikanische Option (mit Konsummöglichkeit bei nicht-optimaler Ausübung).
\end{karte}

\begin{karte}{Ausübung vor/zu/nach \(\tau^*\)}
Ausübung vor \(\tau^*\): \(V_t = p_t(S_t) > h(S_t)\) und \(c_t = 0\). 
Der Verkäufer behält den Gewinn \(V_t - h(S_t)\).

Ausübung zu \(\tau^*\): \(V_{\tau^*} = p_{\tau^*}(S_{\tau^*}) = h(S_{\tau^*})\).
Das Anlagevermögen reicht genau, um den Käufer auszuzahlen. Es bleibt kein Gewinn.

Auszahlung nach \(\tau^*\): Hier kann \(c_{\tau^*} > 0\) konsumiert werden, es startet 
ein neues Stoppproblem mit weiteren Gewinnmöglichkeiten für den Verkäufer.
\end{karte}

\begin{karte}{Put-Option im CRR-Modell}
Für den Preis \(p_t(S_t)\) einer Put-Option gilt: 
\begin{itemize}
    \item \(x\mapsto p_t(x) + x\) ist nicht-fallend für \(t=0,\ldots, T\), 
    \item \(t\mapsto p_t(x)\) ist nicht-wachsend für \(x>0\).
\end{itemize}

Es ist optimal auszuüben, wenn gilt: 
\[ K \geq x + \frac{1}{1+r}(q p_{t+1}(ux) + (1-q)p_{t+1}(dx)). \]
Da der zweite Ausdruck aber nicht-fallend in \(x\) ist, können wir 
\[ x_t^* := \max\set{ x \geq 0: x + \frac{1}{1+r}(q p_{t+1}(ux) + (1-q)p_{t+1}(dx)) \leq K } \]
definieren und erhalten für den optimalen Ausübungszeitpunkt 
\[ \tau^* = \min\set{ t : p_t(S_t) = (K-S_t)^+ } = \min \set{ t: S_t \leq x_t^* } \wedge T. \]

Es gilt \(x_0^* \leq \cdots \leq x_T^* =: K\).
\end{karte}