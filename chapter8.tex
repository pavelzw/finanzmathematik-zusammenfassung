\section{Präferenzen}

\subsection{}
% chapter 8.2
\subsection{Erwarteter Nutzen}

\begin{karte}{Axiome für Präferenzlotterien 1}
Betrachte Lotterien \(L\) mit endl. vielen Auszahlungen \(x_1, \ldots, x_n \in \R\) 
und zugehörigen Wahrscheinlichkeiten \(p_1, \ldots, p_n\). \(L = (x_1, \ldots, x_n; p_1, \ldots, p_n)\).
\begin{description}
    \item[Vergleichbarkeit] Für alle \(x_i, x_j \in \R\) gilt: 
    \[ x_i \succ x_j \Leftrightarrow x_i > x_j, x_j \sim x_i \Leftrightarrow x_i = x_j \]
    \item[Monotonie] Sei \(L_1 = (x_1, x_2; p, 1-p), L_2 = (x_1, x_2; q, 1-q)\). 
    \[ x_1 > x_2 \text{ und } p > q \Rightarrow L_1 \succ L_2. \]
    \item[Transitivität] 
    \[ L_1 \succ L_2, L_2 \succ L_3 \Rightarrow L_1 \succ L_3. \]
    \[ L_1 \sim L_2, L_2 \sim L_3 \Rightarrow L_1 \sim L_3. \]
    \item[Stetigkeit] \[ x_1\succ x_2 \succ x_3 \Rightarrow \exists U(x_2) \in [0,1] \text{ mit } L = (x_1, x_3; U(x_2), 1-U(x_2)) \sim x_2. \]
\end{description}
\end{karte}

\begin{karte}{Axiome für Präferenzlotterien 2}
\begin{description}
    \item[Austauschbarkeit] Sei \(L_1 = (x_1, x_2, x_3; p_1, p_2, p_3)\). 
    \[ x_2 \sim L_2 = (x_1, x_3; q_1, q_2) \Rightarrow L_1 \sim (x_1, L_2, x_3; p_1, p_2, p_3) \]
    \item[Zerlegbarkeit] Sei \(L_i = (x_1, x_2; p_i, 1-p_i)\) (\(i=1,2\)) und \(L = (L_1, L_2; p, 1-p)\), dann gilt: 
    \[ L \sim \tilde{L}(x_1, x_2; p p_1 + (1-p)p_2, p(1-p_1) + (1-p)(1-p_2)). \]
\end{description}
\end{karte}

\begin{karte}{Von-Neumann-Morgenstern-Repräsentation}
Ein Investor akzeptiere die Axiome 1-6. Angenommen, es gibt zwei Lotterien 
\(L_1 = (x_1, \ldots, x_n; p_1, \ldots, p_n)\) und \(L_2 = (x_1, \ldots, x_n; q_1, \ldots, q_n)\). 
Dann gibt es eine wachsende Funktion \(U\), sodass 
\[ L_1 \succ L_2 \Leftrightarrow \sum_{i=1}^n U(x_i) p_i \geq \sum_{i=1}^n U(x_i) q_i. \]

Die Funktion \(U\) spiegelt also die Präferenz des Investors wider. 

Sie liefert eine numerische Repräsentation der Präferenzrelation \(\succ\) 
durch den Wert \(\sum_{i=1}^n p_i U(x_i)\), dieser kann als Erwartungswert der Auszahlungen 
(gewichtet mit \(U\)) interpretiert werden. 

Man bezeichnet \(U\) auch als 
Von-Neumann-Morgenstern-Repräsentation der Präferenzrelation.
\end{karte}

\begin{karte}{Allgemeineres Modell}
Auszahlung ist durch ein Wahrscheinlichkeitsmaß \(\mu\) auf \(S\subset \R\) gegeben. 
\(S\) ist eine zusammenhängende Borelsche Menge. \(\mathcal{P}\) sei eine Menge von Wahrscheinlichkeitsmaßen 
auf \(S\) mit: 
\begin{itemize}
    \item Für alle \(\mu \in \mathcal{P}\) gilt: \(m(\mu) := \int x \mu(dx) \in \R\). 
    \item \(\delta_x \in \mathcal{P}\) für alle \(x\in S\), wobei \(\delta_x\) 
    das Dirac-Maß im Punkt \(x\) ist. 
    \item Die Menge \(\mathcal{P}\) ist konvex, d. h. \(\mu,\nu \in \mathcal{P}\) und 
    \(\alpha \in [0,1] \Rightarrow \alpha\mu + (1-\alpha) \nu \in \mathcal{P}\).
\end{itemize}
\end{karte}

\begin{karte}{Risikoaversion}
Eine Präferenzrelation \(\succ\) auf \(\mathcal{P}\) heißt risikoavers, falls 
\(\delta_{m(\mu)} \succ \mu\) für alle \(\mu \in \mathcal{P}\) mit 
\(\mu \neq \delta_{m(\mu)}\).

Eine Präferenzrelation \(\succ\) auf \(\mathcal{P}\) mit Von-Neumann-Morgenstern-Repräsentation 
\(U(\mu) := \int U(x) \mu(dx)\) ist risikoavers \(\Leftrightarrow U\) ist streng konkav.
\end{karte}

\begin{karte}{Nutzenfunktion}
Eine Funktion \(\abb{U}{S}{\R}\) heißt Nutzenfunktion, falls sie streng wachsend, 
streng konkav und stetig auf \(S\) ist.

Beispiele: 
\begin{itemize}
    \item \(U(x) = -\exp(-\gamma x), \gamma > 0, S := \R\).
    \item \(U(x) = \frac{x^\gamma}{\gamma}, \gamma < 1, S := (0,\infty)\).
    \item \(U(x) = \log x, S := (0,\infty)\).
\end{itemize}
\end{karte}

\begin{karte}{Sicherheitsäquivalent, Risikoprämie}
Sei \(U\) eine Nutzenfunktion. 

Das Sicherheitsäquivalent \(c(\mu)\) einer Lotterie \(\mu\in \mathcal{P}\) ist gegeben durch 
\[ U(c(\mu)) = \int U d\mu, \text{ d. h. } c(\mu) = U^{-1}(\int U d\mu). \]

Die Risikoprämie von \(\mu\) ist definiert als 
\[ p(\mu) := m(\mu) - c(\mu). \]
Da \(U(m(\mu)) = U(\delta_{m(\mu)}) \geq U(\mu) = U(c(\mu))\) gilt \(c(\mu) \leq m(\mu)\) und 
\[ c(\mu) < m(\mu) \Leftrightarrow \mu \neq \delta_{m(\mu)}. \]
\end{karte}

\begin{karte}{Prämie und Arrow-Pratt-Risikoaversionskoeffizient}
Sei \(U\in C^2(S), Z\) ein Risiko und \(x>0\) gegeben. Sei 
\(p(x,Z)\) so, dass 
\[ U(x+\E Z - p(x,Z)) = \E[U(x+Z)]. \]
Sei jetzt \(\E Z = 0\) und \(Var(Z) = \sigma^2\). Taylor liefert: 
\(U(x-p) = U(x) - p U'(x) + O(p^2)\), auf der rechten Seite: 
\[ \E[U(x+Z)] = \E[U(x)+Z U'(x) + Z^2/2 U''(x) + O(Z^3)] = U(x) + \sigma^2/2 U''(x) + o(\sigma^2). \]
Setzen wir nun die beiden Ausdrücke gleich, erhalten wir für die Prämie 
\[ p(x,Z) = \frac{1}{2} \sigma^2 A(x) + o(\sigma^2), \]
wobei 
\[ A(x) = \frac{-U''(x)}{U'(x)} \] 
den absoluten Arrow-Pratt-Risikoaversionskoeffizienten bezeichnet.

Falls \(A(x)\) konstant ist, dann wird \(U\) als CARA-Nutzenfunktion (constant absolute risk aversion)
bezeichnet. Ein Beispiel ist \(U(x) = -e^{-\gamma x}\).
\end{karte}

\begin{karte}{Relative Resikoaversion}
Sei \(\tilde{p}(x,Z)\) eine proportionale Risikoprämie, d. h.: 
\[ \E[U(xZ)] = U(\E[xZ] - x\tilde{p}(x,Z)). \]
Geht man wie beim absoluten Arrow-Pratt-Risikoaversionskoeffizienten vor, so folgt 
\[ \tilde{p}(x,Z) = \frac{1}{2} \sigma^2 R(x) + o(\sigma^2), \]
wobei \(R(x) := xA(x)\) den relativen Arrow-Pratt-Risikoaversionskoeffizienten bezeichnet.

Falls \(R(x)\) konstant ist, heißt \(U\) CRRA-Nutzenfunktion (constant relative risk aversion). 
Beispiele sind die \(\log\)- und Potenznutzenfunktion.
\end{karte}

\begin{karte}{HARA-Nutzenfunktion}
Eine Nutzenfunktion \(\abb{U}{S}{\R}\) heißt HARA-Nutzenfunktion (hyperbolic absolute risk averison), 
falls \(U\in C^2(\R)\) und für Konstanten \(a,b\) gilt 
\[ A(x) = \frac{-U''(x)}{U'(x)} = \frac{1}{ax+b} > 0. \]
\end{karte}

\subsection{Stochastische Dominanz}

\begin{karte}{Stochastische Dominanz erster Ordnung}
Seien \(\mu, \nu \in \mathcal{P}\). Dann dominiert \(\mu\) das Maß \(\nu\) im Sinne der 
stochastischen Dominanz erster Ordnung, falls gilt: 
\[ \int f d\mu \geq \int f d \nu \]
für alle wachsenden Funktionen \(\abb{f}{S}{\R}\), für die die Erwartungswerte 
existieren. Bez. \(\mu \succeq_{FSD} \nu\) (FSD=first order stochastic dominance).
\end{karte}

\begin{karte}{Quantilfunktion, \(\lambda\)-Quantil}
Die Funktion 
\[ F^{-1}(\alpha) := \inf \set{ x\in \R: F(x) \geq \alpha }, \alpha \in (0,1) \]
wird Quantilfunktion der Verteilungsfunktion \(F\) genannt. 

Sei \(\lambda \in (0,1)\). Dann heißt \(q\in \R\) ein \(\lambda\)-Quantil von \(F\), falls gilt: 
\[ F(q) \geq \lambda \text{ und } F(q-) \leq \lambda, \]
wobei \(F(q-)\) der linksseitige Grenzwert von \(F\) an der Stelle \(q\) ist.

\begin{enumerate}
    \item \(F^{-1}\) ist wachsend und linksseitig stetig.
    \item \(F\) ist stetig \(\Leftrightarrow F^{-1}\) ist streng wachsend. 
    \item \(F(x) \geq y \Leftrightarrow x \geq F^{-1}(y)\).
    \item \(F\) stetig \(\Rightarrow F(F^{-1}(x)) = x\).
\end{enumerate}
\end{karte}

\begin{karte}{Charakterisierung Verteilungsfunktion durch Quantilfunktion}
Sei \(X\) eine ZV mit Verteilungsfunktion \(F\) und Quantilfunktion \(F^{-1}\). Weiter 
sei \(U\sim U(0,1)\). Dann gilt: 
\begin{itemize}
    \item \(\P(F^{-1}(U) \leq x) = F(x), x\in \R\), d. h. \(F^{-1}(U) \overset{d}{X}\).
    \item Ist \(F\) stetig, dann ist \(F(X) \sim U(0,1)\).
\end{itemize}
\end{karte}

\begin{karte}{Charakterisierung von FSD}
Seien \(F_\mu(x), F_\nu(x)\) die Verteilungsfunktionen von \(\mu, \nu\) 
und \(F_\mu^{-1}(t), F_\nu^{-1}(t)\) die Quantilfunktionen. 

Für \(\mu, \nu \in \mathcal{P}\) sind äquivalent: 
\begin{itemize}
    \item \(\mu \succeq_{FSD} \nu\).
    \item \(F_\mu(x) \leq F_\nu(x)\) für alle \(x\in \R\).
    \item \(F_{\mu}^{-1}(t) \geq F_\nu^{-1}(t)\) für alle \(t\in (0,1)\).
    \item Es existieren ZV \(X\) und \(Y\) auf \((\Omega, \mathcal{G}, \P)\) 
    mit Verteilungen \(\mu\) und \(\nu\), sodass \(X\geq Y \) \(\P\)-f.s. gilt.
\end{itemize}
\end{karte}

\begin{karte}{Stochastische Dominanz zweiter Ordnung}
Seien \(\mu,\nu \in \mathcal{P}\). Dann dominiert \(\mu\) das Maß \(\nu\) 
im Sinne der stochastischen Dominanz zweiter Ordnung, falls gilt: 
\[ \int U d\mu \geq \int U d\nu, \]
für alle Nutzenfunktionen \(U\) für die die Erwartungswerte existieren. 
Wir schreiben dafür \(\mu \succeq_{SSD} \nu\). 
\end{karte}

\begin{karte}{Charakterisierung stochastische Dominanz zweiter Ordnung}
Für beliebige \(\mu, \nu\in \mathcal{P}\) sind die folgenden Aussagen äquivalent: 
\begin{enumerate}
    \item \(\mu \succeq_{SSD}\nu\).
    \item \(\int f d\mu \geq \int f d\nu\) für alle wachsenden, konkaven Funktionen \(f\).
    \item Für alle \(c\in\R\) gilt \(\int (c-x)^+ \mu(dx) \leq \int (c-x)^+ \nu(dx)\).
    \item Für die Verteilungsfunktionen \(F_\mu\) und \(F_\nu\) von \(\mu, \nu\) gilt: 
    \[ \int_{-\infty}^c F_\mu(x) dx \leq \int_{-\infty}^c F_\nu(x) dx, x\in \R. \]
    \item Für die Quantilfunktionen \(F_\mu^{-1}\) und \(F_\nu^{-1}\) von \(\mu, \nu\) gilt: 
    \[ \int_0^t F_\mu^{-1}(s) ds \geq \int_0^t F_\nu^{-1}(s) ds, 0<t\leq 1. \]
    \item Es existieren Zufallsvariablen \(X\) und \(Y\) auf einem Wahrscheinlichkeitsraum \((\Omega, \mathcal{G}, \P)\)
    mit Verteilungen \(\mu\) und \(\nu\), sodass \(\E[Y|X] \leq X\) \(\P\)-f.s. gilt.
\end{enumerate}
\end{karte}

\begin{karte}{Eigenschaften von SSD}
Sei \(\mu \succeq_{SSD}\nu\). Wähle \(f(x) = x\) als wachsende, konkave Funktion. Dann folgt \(m(\mu) \geq m(\nu)\).

Es gilt: \(\mu \succeq_{FSD} \mu \Rightarrow \mu \succeq_{SSD} \nu\).
\end{karte}