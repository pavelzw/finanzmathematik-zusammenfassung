\section{Erste Begriffe}

\begin{karte}{Derivat}
Ein Derivat ist ein Vertrag, bei dem Zahlungen und Leistungen wesentlich von einer marktbezogenen Referenzgröße 
(Basiswert) abhängen.
\end{karte}

\begin{karte}{Option}
Der Käufer einer Option hat das Wahlrecht ein bestimmtes Finanzgut bis zu einem zukünftigen Zeitpunkt \(T\) 
zu einem vereinbarten Preis \(K\) (strike price, Basispreis, Ausübungspreis) zu kaufen oder zu verkaufen.

Das Kaufrecht wird Call-Option genannt und das Verkaufsrecht wird Put-Option genannt.

Europäische Option: Ausübung ist nur zum Zeitpunkt \(T\) möglich. 
Amerikanische Option: Ausübung ist jederzeit bis zum Zeitpunkt \(T\) möglich.
\end{karte}

\begin{karte}{No-arbitrage-Prinzip, Optionspreis}
Es darf keine Arbitrage (risikoloser Gewinn) möglich sein. Die Auszahlung \(H\) wird mit anderen 
Finanzinstrumenten repliziert. Das Anfangskapital, das nötig ist, um \(H\) zu replizieren, ist der Preis der Option.
\end{karte}