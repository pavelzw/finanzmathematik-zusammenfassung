\section{Risikoneutrale Bewertung}

\begin{karte}{Risikoneutrale Bewertung}
\begin{itemize}
    \item Voraussetzung: Finanzmarkt ist arbitragefrei.
    \item \(\F_0 = \set{\emptyset, \Omega}\).
    \item \(\F_T = \F =\) Potenzmenge von \(\Omega\).
    \item Preis \(\pi(H)\) zur Zeit \(t=0\) eines erreichbaren Zahlungsanspruchs \(H\) 
    ist gerade das Anfangsvermögen einer Hedging-Strategie. 
    \item Ist \(\varphi\) eine Hedging-Strategie für \(H\), so gilt 
    \(V_t^\varphi=\) Preis von \(H\) zur Zeit \(t\).
\end{itemize}
\end{karte}

\begin{karte}{Eindeutigkeit des Preises}
Für jeden erreichbaren Zahlungsanspruch \(H\) 
und für selbstfinanzierende Strategien \(\varphi\) und \(\psi\) 
mit \(V_T^\varphi = H = V_T^\psi\) gilt: 
\[ V_t^\varphi = V_t^\psi. \]
\(\pi_t(H) := V_t^\varphi\) ist also der (eindeutige) Preis des 
Zahlungsanspruchs \(H\) zur Zeit \(t\).
\end{karte}

\begin{karte}{Risikoneutrale Bewertungsformel}
Sei \(H\) ein erreichbarer Zahlungsanspruch. Dann gilt für den Preis von \(H\)
zur Zeit \(t\) 
\[ \pi_t(H) = B_t \E_\Q\left[ \frac{H}{B_T} | \F_t \right] \]
für alle \(\Q\in \M^*\). Insbesondere ist \(\pi_t(H)\) unabhängig von der Wahl von 
\(\Q\in \M^*\).
\end{karte}

\begin{karte}{Preis zur Zeit \(t=0\)}
Sei \(H\) ein erreichbarer Zahlungsanspruch. Dann gilt für den Preis von \(H\)
zur Zeit \(t=0\):
\[ \pi(H) = \E_\Q \left[ \frac{H}{B_t} \right] \]
für alle \(\Q\in\M^*\).
\end{karte}

\begin{karte}{Put-Call-Parität}
Es gilt die sogenannte Put-Call-Parität: 
\[ C + KB_T^{-1} = P + S_0. \]
Allgemeiner gilt zur Zeit \(t\):
\[ C_t + K \frac{B_t}{B_T} = P_t + S_t. \]
\end{karte}

\begin{karte}{Symmetrische Irrfahrt}
Eine Folge von ZV \((Z_t)\) mit \(Z_0 := 0\) und \(Z_t := \sum_{k=1}^t Y_k\), 
wobei \(Y_1, \ldots\) u.i.v. ZV sind mit \(\P(Y_k = 1) = \P(Y_k = -1) = \frac{1}{2}\) heißt symmetrische Irrfahrt auf \(\Z\).

Für \(T\in \N\) und \(b\in\N\) mit \(-T\leq b\leq T\) gilt: 
\[ \P(Z_T = b) = \begin{cases}
    (\frac{1}{2})^T \binom{T}{(T+b)/2}, &\text{ falls} T, b \text{ die gleiche Parität besitzen.} \\
    0, & \text{ sonst}.
\end{cases} \]
Gleiche Parität bedeutet, dass beide Zahlen gerade oder beide ungerade sind.
\end{karte}

\begin{karte}{Spiegelungsprinzip}
Sei \(M_t := \max\set{ Z_0, \ldots, Z_t }\) das laufende Maximum. 

Für \(k\in\N\) und \(m\in\N_0\) gilt: 
\[ \P(M_T\geq k, Z_T = k-m) = \P(Z_T = k+m). \]
\end{karte}

\begin{karte}{Hilfslemma für Irrfahrten}
Unter \(\Q\) gilt für die Irrfahrt \((Z_t)\) auf \(\Z\) und \(M_t := \max\set{Z_0,\ldots, Z_t}\) 
für \(k\in\N\) und \(m\in\N_0\), falls \(T\) und \(k+m\) die gleiche Parität besitzen, 
\[ \Q(M_T \geq k, Z_T = k-m) = \binom{T}{(T+k+m)/2} q^{(T+k-m)/2} (1-q)^{(T-k+m)/2}, \]
\[ \Q(M_T = k, Z_T = k-m) = \binom{T+1}{(T-k-m)/2} \frac{k+m+1}{T+1} q^{(T+k-m)/2} (1-q)^{(T-k+m)/2}. \]
\end{karte}
