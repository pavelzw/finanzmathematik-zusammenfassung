\section{Endliche Finanzmärkte}

\begin{karte}{Stochastischer Prozess, Filtration, adaptiert}
\begin{enumerate}
    \item Eine Familie \((X_t)_{t\in I}\) von ZVen \(\abb{X_t}{\Omega}{\R}\) heißt stochastischer Prozess.
    \item Eine Familie \((\mathcal{F}_t)_{t\in I}\) von Teil-\(\sigma\)-Algebren \(\mathcal{F}_t \subset \mathcal{F}\) heißt Filtration, 
    falls für alle \(s < t, s,t\in I\) gilt \(\mathcal{F}_s \subset \mathcal{F}_t\).
    \item Ein stochastischer Prozess \((X_t)_{t\in I}\) heißt adaptiert bezüglich der Filtration \((\mathcal{F}_t)_{t\in I}\), 
    falls \(X_t\) \(\mathcal{F}_t\)-messbar ist für jedes \(t\in I\).
\end{enumerate}

Sei \((S_t)_{t\in I}\) ein stochastischer Prozess, der den Preisverlauf einer Aktie darstellt. 
Dann ist \(\mathcal{F}_t^S = \sigma(S_0, \ldots, S_t) = \set{ (S_0, \ldots, S_t)^{-1}(B) \;|\; B \subset \R^{t+1} \text{ messbar} }\)
eine Filtration von \(\mathcal{F}_T =: \mathcal{F}\) (auch natürliche Filtration genannt). \((S_t)\) ist adaptiert bzgl. \((\mathcal{F}_t^S)\).
\(\mathcal{F}_t^S\) gibt die Information wieder, die bis zum Zeitpunkt \(t\) vorliegt.
\end{karte}

\subsection{Finanzmarkt}

\begin{karte}{Finanzmarkt}
\begin{itemize}
    \item \(\Omega\) ist endlich.
    \item \(\P(\set{\omega}) > 0 \;\forall \omega \in \Omega\).
    \item \(\mathcal{F}\) ist die Potenzmenge von \(\Omega\).
    \item Filtration \((\mathcal{F}_t)_{t\in I}\) gegeben. 
    \item \(\mathcal{F}_0 = \set{\emptyset, \Omega}\) ist die triviale \(\sigma\)-Algebra und \(\mathcal{F}_T = \mathcal{F}\).
\end{itemize}
Der Finanzmarkt besteht aus \(d+1\) Anlagemöglichkeiten: 
\begin{itemize}
    \item ein risikoloses Wertpapier mit deterministischem Preisprozess \((B_t) = (B_0, \ldots, B_T), B_0 = 1, B_{t+1} \geq B_t > 0\).
    \item \(d\) risikobehaftete Wertpapiere mit stochastischen Preisprozessen \((S_t^k) = (S_0^k, \ldots, S_T^k), S_t^k(\omega) > 0\). 
    Setze \(S_t := (S_t^1, \ldots, S_t^d)\). Die Prozesse \(S_t^k\) seien adaptiert bzgl. der gegebenen Filtration \((\mathcal{F}_t)\).
\end{itemize}
\end{karte}

\begin{karte}{Handelsstrategie}
Eine Handelsstrategie oder ein Portfolio \(\varphi = (\varphi_0, \ldots, \varphi_{T-1})\) ist ein 
\(\R^{d+1}\)-wertiger, \((\mathcal{F}_t)\)-adaptierter stochastischer Prozess, d. h. \(\varphi_t\) 
ist \(\F_t\)-messbar und 
\[ \varphi_t := (\alpha_t, \beta_t). \]
Dabei ist \(\beta_t\) die Stückzahl des risikolosen Wertpapiers in \([t,t+1)\) 
und \(\alpha_t := (\alpha_t^1, \ldots, \alpha_t^d)\) die Stückzahlen der risikobehafteten Wertpapiere 
in \([t,t+1)\). 
\end{karte}

\begin{karte}{Faktorisierungssatz}
Sei \((\Omega, \F)\) ein Messraum und \(\abb{X,Y}{\Omega}{\R}\) ZVen. Dann sind folgende Aussagen äquivalent: 
\begin{itemize}
    \item \(X\) ist \((\sigma(Y), \mathcal{B})\)-messbar. 
    \item Es existiert eine messbare Abbildung \(\abb{h}{\R}{\R}\) mit \(X = h(Y)\).
\end{itemize}

Ist eine Handelsstrategie \((\F_t^S)\)-adaptiert, bedeutet dies nach dem Faktorisierungssatz, 
dass \(\beta_t = \beta_t(S_0, \ldots, S_t)\) und \(\alpha_t^k = \alpha_t^k(S_0, \ldots, S_t)\).
\end{karte}

\begin{karte}{Wert einer Handelsstrategie}
Der Wert der Handelsstrategie \(\varphi\) zum Zeitpunkt \(t = 0, \ldots, T-1\) ist gegeben durch 
\[ V_t^\varphi := \beta_t B_t + \sum_{k=1}^d \alpha_t^k S_t^k =: \beta_t B_t + \alpha_t \cdot S_t. \]
Außerdem setzen wir \(V_T^\varphi := \beta_{T-1}B_T + \alpha_{T-1} \cdot S_T\).
\end{karte}

\begin{karte}{Selbstfinanzierend}
Eine Handelsstrategie \(\varphi\) heißt selbstfinanzierend, falls 
\[ \beta_{t-1} B_t + \alpha_{t-1} \cdot S_t = \beta_t B_t + \alpha_t \cdot S_t \]
für alle \(t=1,\ldots, T-1\).
\end{karte}

\begin{karte}{Definition \(\A\)}
Es sei 
\[ \A := \set{ \alpha : \alpha_t \text{ ist } \F_t \text{-messbar}, t=0,\ldots, T-1 } \]
die Menge der Investitionsstrategien in die risikobehafteten Wertpapiere. Bei vorgegebenem Anfangsvermögen 
kann jedes \(\alpha \in \A\) selbstfinanzierend ergänzt werden.
\end{karte}

\begin{karte}{Bestimmung von \(\beta\) aus \(\alpha\) und \(V_0\)}
Sei \(\Delta X_t := X_t - X_{t-1}\). Weiter sei \(\Delta \alpha_t := (\Delta \alpha_t^1, \ldots, \Delta \alpha_t^d)\). 

Sei \(\varphi\) eine selbstfinanzierende Handelsstrategie. Dann gilt 
\[ \beta_t = \beta_0 - \sum_{n=1}^t \Delta \alpha_n \cdot \frac{S_n}{B_n} = V_0^\varphi - \sum_{n=0}^t \Delta \alpha_n \cdot \frac{S_n}{B_n}, \]
wobei \(\Delta \alpha_0^k := \alpha_0^k\).
\end{karte}

\begin{karte}{Diskontierter Preis}
Es ist 
\[ \tilde{S}_t^k := \frac{S_t^k}{B_t} \]
der diskontierte Preis des \(k\)-ten risikobehafteten Wertpapiers zur Zeit \(t\).
\end{karte}

\begin{karte}{Diskontierter Wert einer Handelsstrategie}
Sei \(\varphi\) eine selbstfinanzierende Handelsstrategie. Dann gilt 
\[ \frac{V_t^\varphi}{B_t} = V_0^\varphi + \sum_{n=1}^t \alpha_{n-1} \cdot \Delta \tilde{S}_n. \]
\end{karte}

\begin{karte}{Gewinnprozess}
Den Prozess \((G_t^\alpha)\) für \(\alpha \in \A\), der durch \(G_0^\alpha := 0\) und 
\[ G_t^\alpha := \sum_{n=1}^t \alpha_{n-1} \cdot \Delta \tilde{S}_n \]
definiert ist, bezeichnen wir als Gewinnprozess von \(\alpha\). Es gilt dann für eine selbstfinanzierende 
Handelsstrategie \(\varphi = (\alpha, \beta)\): 
\[ \frac{V_t^\varphi}{B_t} = V_0^\varphi + G_t^\alpha. \]
\end{karte}

\begin{karte}{Arbitragestrategie}
Eine selbstfinanzierende Handelsstrategie \(\varphi\) wird als Arbitragestrategie bezeichnet, falls 
\[ V_0^\varphi = 0,\quad \P(V_T^\varphi \geq 0) = 1 \text{ und } \P(V_T^\varphi > 0) > 0. \]
Wir sagen, dass eine Arbitragemöglichkeit existiert, falls eine Arbitragestrategie existiert. 
(NA) bedeutet, dass keine Arbitragemöglichkeit gegeben ist.
\end{karte}

\begin{karte}{Arbitragestrategie Charakterisierung}
Die folgenden Aussagen sind äquivalent:
\begin{itemize}
    \item Es gibt eine Arbitragestrategie.
    \item Es gibt ein \(t\in \set{1,\ldots, T}\) und einen \(\F_{t-1}\)-messbaren Zufallsvektor \(\abb{\eta}{\Omega}{\R^d}\), sodass 
    \[ \P(\eta \cdot (\tilde{S}_t - \tilde{S}_{t-1}) \geq 0) = 1 \text{ und } \P(\eta \cdot (\tilde{S}_t - \tilde{S}_{t-1}) > 0) > 0. \]
\end{itemize}
\end{karte}

\subsection{Optionen}

\begin{karte}{Zahlungsanspruch}
Ein Zahlungsanspruch ist eine \(\F_T\)-messbare Zufallsvariable \(H\) mit Werten in \(\R\).

Ist \(H\) sogar \(\F_T^S\)-messbar, so ist nach dem Faktorisierungssatz \(H = h(S_0, \ldots, S_T)\) 
für eine Funktion \(h\).

Beispiele: 
\begin{itemize}
    \item Europäische Call-Option mit Basispreis \(K\): \(H = (S_T - K)^+\).
    \item Europäische Put-Option mit Basispreis \(K\): \(H = (K - S_T)^+\).
\end{itemize}
\end{karte}

\begin{karte}{Vollständiger Markt}
Ein Zahlungsanspruch \(H\) heißt erreichbar, wenn es eine selbstfinanzierende Handelsstrategie \(\varphi\) 
gibt mit \(V_T^\varphi = H\). Dann heißt \(\pi(H) := V_0^\varphi\) ein Preis von \(H\) und \(\varphi\) eine 
Hedging-Strategie von \(H\).

Ein Markt wird als vollständig bezeichnet, wenn jeder Zahlungsanspruch erreichbar ist.
\end{karte}

\begin{karte}{Eindeutigkeit \(\pi(H)\)}
Es gelte (NA). Dann ist der Preis \(\pi(H)\) für erreichbare Zahlungsansprüche eindeutig bestimmt und damit 
unabhängig von der Wahl der Hedging-Strategie.
\end{karte}